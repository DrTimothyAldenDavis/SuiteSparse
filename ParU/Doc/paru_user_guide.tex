\documentclass[12pt]{article}
%\batchmode
\usepackage{hyperref}
\usepackage{minted}

\topmargin -0.5in
\textheight 9.0in
\oddsidemargin 0pt
\evensidemargin 0pt
\textwidth 6.5in

%-------------------------------------------------------------------------------
% get epsf.tex file, for encapsulated postscript files:
\input epsf
%-------------------------------------------------------------------------------
% macro for Postscript figures the easy way
% usage:  \postscript{file.ps}{scale}
% where scale is 1.0 for 100%, 0.5 for 50% reduction, etc.
%
\newcommand{\postscript}[2]
{\setlength{\epsfxsize}{#2\hsize}
\centerline{\epsfbox{#1}}}
%-------------------------------------------------------------------------------

\title{User's Guide for ParU, an unsymmetric multifrontal multithreaded sparse
LU factorization package}
\author{Mohsen Aznaveh\thanks{email: aznaveh@tamu.edu.},
Timothy A. Davis\thanks{email: DrTimothyAldenDavis@gmail.com,
http://www.suitesparse.com.}}

% version of SuiteSparse/ParU
\date{VERSION 0.2.0, Aug 1, 2024}


%-------------------------------------------------------------------------------
\begin{document}
%-------------------------------------------------------------------------------
\maketitle

\begin{abstract}

ParU is an implementation of the multifrontal sparse LU factorization method.
Parallelism is exploited both in the BLAS and across different frontal matrices
using OpenMP tasking, a shared-memory programming model for modern multicore
architectures. The package is written in C++ and real sparse matrices are
supported.

ParU, Copyright (c) 2022-2024, Mohsen Aznaveh and Timothy A. Davis,
All Rights Reserved.
SPDX-License-Identifier: GPL-3.0-or-later
\end{abstract}

\newpage
{\small
\tableofcontents
}

%-------------------------------------------------------------------------------
\section{Introduction}
\label{intro}
%-------------------------------------------------------------------------------

The algorithms used in ParU are discussed in \cite{AznavehThesis},
a copy of which is in the \verb'ParU/Doc' folder.
% FUTURE: add the ACM TOMS submission to the Doc folder.
This document gives detailed information on the installation and use of ParU.
ParU is a parallel sparse direct solver that uses OpenMP tasking for
parallelism. ParU calls UMFPACK for the symbolic analysis phase, after that,
some symbolic analysis is done by ParU itself, and then the numeric phase
starts. The numeric computation is a task parallel phase using OpenMP, and each
task calls parallel BLAS; i.e. nested parallelism.  The performance of BLAS has
a heavy impact on the performance of ParU. Moreover, the way parallel BLAS can
be called in a nested environment can also be very important for ParU's
performance.

%-------------------------------------------------------------------------------
\section{Using ParU in C and C++}
%-------------------------------------------------------------------------------

ParU relies on CHOLMOD for its basic sparse matrix data structure, a compressed
sparse column format.  CHOLMOD provides interfaces to the AMD, COLAMD, and
METIS ordering methods and many other functions. ParU also relies on UMFPACK
for its symbolic analysis.

%-------------------------------------------------------------------------------
\subsection{Installing the C/C++ library on any system}
%-------------------------------------------------------------------------------

All of SuiteSparse can be built by \verb'cmake' with a single top-level
\verb'CMakeLists.txt' file.  In addition, each package (including ParU) has its
own \verb'CMakeLists.txt' file to build that package individually.  This is the
simplest method for building ParU and its dependent pacakges on all systems.

%-------------------------------------------------------------------------------
\subsection{Installing the C/C++ library on Linux/Unix}
%-------------------------------------------------------------------------------

In Linux/MacOs, type \verb'make' at the command line in either the
\verb'SuiteSparse' directory (which compiles all of SuiteSparse) or in the
\verb'SuiteSparse/ParU' directory (which just compiles ParU).  ParU will be
compiled; you can type \verb'make demos' to run a set of simple demos.

The use of \verb'make' is optional.  The top-level \verb'ParU/Makefile' is
a simple wrapper that uses \verb'cmake' to do the actual build.

To fully test the coverage of the lines ParU, go to the \verb'Tcov'
directory and type \verb'make'.  This test requires Linux.

To install the shared library (by default, into \verb'/usr/local/lib' and
\verb'/usr/local/include'), do {\tt make install}.  To uninstall, do
{\tt make uninstall}.  For more options, see the {\tt ParU/README.md} file.

%-------------------------------------------------------------------------------
\subsection{C/C++ Example}
%-------------------------------------------------------------------------------

Below is a simple C++ program that illustrates the use of ParU.  The program
reads in a problem from \verb'stdin' in MatrixMarket format
\cite{BoisvertPozoRemingtonBarrettDongarra97}, solves it, and prints the norm
of \verb'A' and the residual.  Some error testing code is omited to simplify
the program, but a robust user application should check the return values from
ParU.  The full program can be found in \verb'ParU/Demo/paru_simple.cpp'.  Note
that ParU supports only real double-precision matrices.

Refer to the CHOLMOD User guide for the CHOLMOD methods used below.

\begin{minted}{cpp}
#include <iostream>
#include <iomanip>
#include <ios>
#include <cmath>
#include "ParU.h"

int main(int argc, char **argv)
{
    cholmod_common Common, *cc;
    cholmod_sparse *A = NULL;
    ParU_Symbolic Sym = NULL;
    ParU_Numeric Num = NULL;
    ParU_Control Control = NULL;
    double *b = NULL, *x = NULL;

    //~~~~~~~~~Reading the input matrix ~~~~~~~~~~~~~~~~~~~~~~~~~~~~~~~~~~~~~~~~
    // start CHOLMOD
    cc = &Common;
    int mtype;
    cholmod_l_start(cc);
    A = (cholmod_sparse *)cholmod_l_read_matrix(stdin, 1, &mtype, cc);

    //~~~~~~~~~~~~~~~~~~~Starting computation~~~~~~~~~~~~~~~~~~~~~~~~~~~~~~~~~~
    std::cout << "================= ParU, a simple demo: ===================\n";
    ParU_Info info;
    ParU_Analyze(A, &Sym, Control);
    int64_t n, anz;
    ParU_Get (Sym, Num, PARU_GET_N, &n, Control));
    ParU_Get (Sym, Num, PARU_GET_ANZ, &anz, Control));
    std::cout << "Input matrix is " << n << "x" << n <<
        " nnz = " << anz << std::endl;
    ParU_Factorize(A, Sym, &Num, Control);
    std::cout << "ParU: factorization was successful." << std::endl;

    //~~~~~~~~~~~~~~~~~~~ Computing the residual, norm(b-Ax) ~~~~~~~~~~~~~~~~~~~
    b = (double *)malloc(n * sizeof(double));
    x = (double *)malloc(n * sizeof(double));
    for (int64_t i = 0; i < n; ++i) b[i] = i + 1;
    ParU_Solve(Sym, Num, b, x, Control));
    double resid, anorm, xnorm, rcond;
    ParU_Residual(A, x, b, resid, anorm, xnorm, Control));
    ParU_Get (Sym, Num, PARU_GET_RCOND_ESTIMATE, &rcond, Control);
    double rresid = (anorm == 0 || xnorm == 0 ) ? 0 : (resid/(anorm*xnorm));
    std::cout << std::scientific << std::setprecision(2)
        << "Relative residual is |" << rresid << "| anorm is " << anorm
        << ", xnorm is " << xnorm << " and rcond is " << rcond << "."
        << std::endl;

    //~~~~~~~~~~~~~~~~~~~End computation~~~~~~~~~~~~~~~~~~~~~~~~~~~~~~~~~~~~~~~
    free(b);
    free(x);
    ParU_FreeNumeric(&Num, Control);
    ParU_FreeSymbolic(&Sym, Control);
    ParU_FreeControl(&Control);
    cholmod_l_free_sparse(&A, cc);
    cholmod_l_finish(cc);
    return (info);
}
\end{minted}

A simple demo for the C interface is shown next. You can see the complete demo
in \verb'ParU/Demo/paru_simplec.c'.

\begin{minted}{c}
#include "ParU.h"
int main(int argc, char **argv)
{
    cholmod_common Common, *cc = NULL;
    cholmod_sparse *A = NULL;
    ParU_C_Symbolic Sym = NULL;
    ParU_C_Numeric Num = NULL;
    ParU_C_Control Control = NULL;
    double *b = NULL, *x = NULL;

    //~~~~~~~~~Reading the input matrix ~~~~~~~~~~~~~~~~~~~~~~~~~~~~~~~~~~~~~~~~
    // start CHOLMOD
    cc = &Common;
    int mtype;
    cholmod_l_start(cc);
    // read in the sparse matrix A from stdin
    A = (cholmod_sparse *)cholmod_l_read_matrix(stdin, 1, &mtype, cc);

    //~~~~~~~~~~~~~~~~~~~Starting computation~~~~~~~~~~~~~~~~~~~~~~~~~~~~~~~~~~
    ParU_C_Analyze(A, &Sym, Control);
    int64_t n, anz;
    ParU_C_Get_INT64 (Sym, Num, PARU_GET_N, &n, Control);
    ParU_C_Get_INT64 (Sym, Num, PARU_GET_ANZ, &anz, Control);
    printf("Input matrix is %" PRId64 "x%" PRId64 " nnz = %" PRId64 " \n",
        n, n, anz);
    ParU_C_Factorize(A, Sym, &Num, Control);

    //~~~~~~~~~~~~~~~~~~~ Computing the residual, norm(b-Ax) ~~~~~~~~~~~~~~~~~~~
    b = (double *)malloc(n * sizeof(double));
    x = (double *)malloc(n * sizeof(double));
    for (int64_t i = 0; i < n; ++i) b[i] = i + 1;
    ParU_C_Solve_Axb(Sym, Num, b, x, Control);
    double resid, anorm, xnorm;
    ParU_C_Residual_bAx(A, x, b, &resid, &anorm, &xnorm, Control);
    double rresid = (anorm == 0 || xnorm == 0 ) ? 0 : (resid/(anorm*xnorm));
    double rcond;
    ParU_C_Get_FP64(Sym, Num, PARU_GET_RCOND_ESTIMATE, &rcond, Control);
    printf("Relative residual is |%.2e|, anorm is %.2e, xnorm is %.2e, "
        " and rcond is %.2e.\n",
        rresid, anorm, xnorm, rcond);

    //~~~~~~~~~~~~~~~~~~~End computation~~~~~~~~~~~~~~~~~~~~~~~~~~~~~~~~~~~~~~~
    if (b != NULL) free(b);
    if (x != NULL) free(x);
    ParU_C_FreeNumeric(&Num, Control);
    ParU_C_FreeSymbolic(&Sym, Control);
    ParU_C_FreeControl(&Control);
    cholmod_l_free_sparse(&A, cc);
    cholmod_l_finish(cc);
    return (info);
}
\end{minted}

%-------------------------------------------------------------------------------
\subsection{{\sf ParU\_Info}: return values of each ParU method}
%-------------------------------------------------------------------------------

    All ParU C and C++ routines return an enum of type \verb'ParU_Info'.  The
    user application should check this return value before continuing.

    {\footnotesize
    \begin{verbatim}
    typedef enum ParU_Info
    {
        PARU_SUCCESS = 0,           // everying is fine
        PARU_OUT_OF_MEMORY = -1,    // ParU ran out of memory
        PARU_INVALID = -2,          // inputs are invalid (NULL, for example)
        PARU_SINGULAR = -3,         // matrix is numerically singular
        PARU_TOO_LARGE = -4         // problem too large for the BLAS
    } ParU_Info ; \end{verbatim}}

%-------------------------------------------------------------------------------
\section{C++ Syntax}
%-------------------------------------------------------------------------------

%-------------------------------------------------------------------------------
\subsection{{\sf ParU\_Version}: version of the ParU package}
%-------------------------------------------------------------------------------

    ParU has two mechanisms for informing the user application of its date and
    version: macros that are \verb'#define'd in \verb'ParU.h', and a
    \verb'ParU_Version' function.  Both methods are provided since it's
    possible that the \verb'ParU.h' header found when a user application was
    compiled might not match the same version found when the same user
    application was linked with the compiled ParU library.

    {\footnotesize
    \begin{verbatim}
    #define PARU_DATE "June 20, 2024"
    #define PARU_VERSION_MAJOR  1
    #define PARU_VERSION_MINOR  0
    #define PARU_VERSION_UPDATE 0
    ParU_Info ParU_Version (int ver [3], char date [128]) ; \end{verbatim}}

    \verb'ParU_Version' returns the version in \verb'ver' array (major, minor,
    and update, in that order), and the date in the \verb'date' array provided
    by the user application.

%-------------------------------------------------------------------------------
\subsection{{\sf ParU\_Control}: parameters that control ParU}
\label{control}
%-------------------------------------------------------------------------------

    The \verb'ParU_Control' structure contains parameters that control various
    ParU options.  The object is created by \verb'ParU_InitControl',
    modified by \verb'ParU_Set', and deleted by \verb'ParU_FreeControl'.
    Its contents can be queried with \verb'ParU_Get'.

    Any ParU function can be passed a NULL pointer for its \verb'Control'
    parameter.  In that case, defaults are used.  To use non-default
    parameters, create a \verb'Control' object and then set its parameters.

    {\footnotesize
    \begin{verbatim}
    ParU_Info ParU_InitControl
    (
        // output:
        ParU_Control *Control_handle    // Control object to create
    ) ; \end{verbatim}}

    {\footnotesize
    \begin{verbatim}
    ParU_Info ParU_FreeControl
    (
        // input/output:
        ParU_Control *Control_handle    // Control object to free
    ) ; \end{verbatim}}

%-------------------------------------------------------------------------------
\subsection{{\sf ParU\_Set}: set parameters in the Control object}
\label{set}
%-------------------------------------------------------------------------------

There are four variants of \verb'ParU_Set' for setting control parameters, two
for integers (\verb'int64_t' and \verb'int32_t') and two for floating-point
(\verb'double' and \verb'float') which are the valid options for \verb'type' in
the signature below.

    {\footnotesize
    \begin{verbatim}
    ParU_Info ParU_Set
    (
        // input
        ParU_Control_enum parameter,    // parameter to set
        type c,                         // value to set it to
        // control:
        ParU_Control Control
    ) ; \end{verbatim}}

The \verb'ParU_Control_enum parameter' defines which parameter to set,
described below.

    {\footnotesize
    \begin{verbatim}
    // enum for ParU_Set/ParU_Get for Control object
    typedef enum
    {

        // int64_t parameters for ParU_Set and ParU_Get:
        PARU_CONTROL_MAX_THREADS = 1001,          // max number of threads
        PARU_CONTROL_STRATEGY = 1002,             // ParU strategy
        PARU_CONTROL_UMFPACK_STRATEGY = 1003,     // UMFPACK strategy
        PARU_CONTROL_ORDERING = 1004,             // UMFPACK ordering
        PARU_CONTROL_RELAXED_AMALGAMATION = 1005, // goal for # pivots in fronts
        PARU_CONTROL_PANEL_WIDTH = 1006,          // # of pivots in a panel
        PARU_CONTROL_DGEMM_TINY = 1007,           // dimension of tiny dgemm's
        PARU_CONTROL_DGEMM_TASKED = 1008,         // dimension of tasked dgemm's
        PARU_CONTROL_DTRSM_TASKED = 1009,         // dimension of tasked dtrsm's
        PARU_CONTROL_PRESCALE = 1010,             // prescale input matrix
        PARU_CONTROL_SINGLETONS = 1011,           // filter singletons, or not
        PARU_CONTROL_MEM_CHUNK = 1012,            // chunk size of memset and memcpy

        // int64_t parameter, for ParU_Get only:
        PARU_CONTROL_OPENMP = 1013,               // if ParU compiled with OpenMP;
                                                  // (for ParU_Get only, not set)

        // double parameters for ParU_Set and ParU_Get:
        PARU_CONTROL_PIVOT_TOLERANCE = 2001,      // pivot tolerance
        PARU_CONTROL_DIAG_PIVOT_TOLERANCE = 2002, // diagonal pivot tolerance

        // pointer to const string (const char **), for ParU_Get only:
        PARU_CONTROL_BLAS_LIBRARY_NAME = 3001,    // BLAS library used
        PARU_CONTROL_FRONT_TREE_TASKING = 3002,   // parallel or sequential

    }
    ParU_Control_enum ; \end{verbatim}}

    For integer parameters:

    \begin{itemize}
    \item \verb'PARU_CONTROL_MAX_THREADS': % max number of threads
        number of OpenMP threads to use.  If zero or negative, the value is
        obtained from \verb'omp_get_max_threads'.

    \item \verb'PARU_CONTROL_STRATEGY': % ParU strategy
        Ordering and factorization strategy to use.

        \begin{itemize}
        \item \verb'PARU_STRATEGY_AUTO':  ParU selects its strategy
            automatically, based on the symbolic analysis of the input matrix,
            by selecting whichever strategy that UMFPACK selects.

        \item \verb'PARU_STRATEGY_UNSYMMETRIC':
            % ParU will order columns of the matrix $A'A$ via COLAMD or METIS.
            During numerical factorization, no preference is
            given for diagonal entries when looking for pivots.
        \item \verb'PARU_STRATEGY_SYMMETRIC':
            % ParU will order the columns of the matrix $A+A'$ via AMD or METIS.
            During numerical factorization, diagonal entries are given
            preference when looking for pivots.  This strategy works well when
            the nonzero pattern of the matrix is mostly symmetric, and when the
            diagonal of the matrix is mostly zero-free.

        \end{itemize}

    \item \verb'PARU_CONTROL_UMFPACK_STRATEGY': % UMFPACK strategy
        The ordering strategy used by UMFPACK.  ParU uses UMFPACK for its
        ordering and symbolic analysis phases.  The ParU and UMFPACK strategies
        are normally the same, but there are cases where best performance is
        obtained with different strategies.

        \begin{itemize}
        \item \verb'UMFPACK_STRATEGY_AUTO':  UMFPACK selects its strategy
            automatically, based on the symbolic analysis of the input matrix.
            Let $S$ be the matrix found by UMFPACK after it removes the row and
            column {\em singletons} (defined below).  If the singleton removal
            preserves the diagonal of $A$, the nonzero pattern of $S$ has a
            {\em symmetry} $\sigma \ge 0.3$, and the diagonal of $S$ is at
            least 90\% nonzero, then the symmetric strategy is chosen.
            Otherwise, the unsymmetric strategy is chosen.

            The {\em symmetry} $\sigma$ of ${S}$ is defined as the number of
            {\em matched} off-diagonal entries, divided by the total number of
            off-diagonal entries.  An entry $s_{ij}$ is matched if $s_{ji}$ is
            also an entry.  They need not be numerically equal.  An {\em entry}
            is a value in $A$ which is present in the input data structure.
            All nonzeros are entries, but some entries may be numerically zero.

            A {\em row singleton} is an entry $a_{ij}$ with a single entry in
            the $i$th row of the matrix $A$.  A {\em column singleton} is an
            entry $a_{ij}$ with a single entry in the $j$th column of the
            matrix $A$.  When a singleton $a_{ij}$ is found, row $i$ and column
            $j$ are removed and the process repeats.  In the final pruned
            matrix, all rows and columns have at least two entries.

        \item \verb'UMFPACK_STRATEGY_UNSYMMETRIC': UMFPACK will order columns
            of the matrix $A'A$ via COLAMD or METIS.

        \item \verb'UMFPACK_STRATEGY_SYMMETRIC': UMFPACK will order the columns
            of the matrix $A+A'$ via AMD or METIS.

        \end{itemize}

    \item \verb'PARU_CONTROL_ORDERING': % UMFPACK ordering

        The default ordering is \verb'PARU_ORDERING_METIS_GUARD', which provides
        low fill-in.  However, this ordering can be costly to compute.  It is
        best suited to the case when multiple matrices with the same nonzero
        pattern are being factorized, where the symbolic analysis is just
        performed once, and reused for each of the subsequent numerical
        factorizations.

        For a one-off factorization of a single matrix,
        \verb'PARU_ORDERING_AMD' can be faster; the ordering is much faster to
        compute than METIS, and the quality of the ordering (which determines
        the fill-in and flop count in the numerical factorization) can often be
        acceptable.  This ordering option uses AMD when using the symmetric
        strategy, or COLAMD when using the unsymmetric strateg.

        \begin{itemize}
        \item \verb'PARU_ORDERING_AMD':
            use AMD on $A+A'$ (symmetric strategy) or COLAMD (unsymmetric
            strategy), which orders $A'A$ without forming it explicitly.
        \item \verb'PARU_ORDERING_METIS': use METIS on $A+A'$ (symmetric
            strategy) or $A'A$ (unsymmetric strategy), where $A'A$ is
            explicitly formed.
        \item \verb'PARU_ORDERING_METIS_GUARD':
            use METIS, AMD, or COLAMD.  This is the default.
            Symmetric strategy: always use METIS on
            $A+A'$.  Unsymmetric strategy: use METIS on $A'A$, unless $A$ has
            one or more rows with $3.2\sqrt{n}$ or more entries.  In that case,
            $A'A$ is very costly to form, and COLAMD is used instead of METIS.
        \item \verb'PARU_ORDERING_CHOLMOD':
            use CHOLMOD (AMD/COLAMD then METIS, see above).
        \item \verb'PARU_ORDERING_BEST':
            try many orderings and pick the best one found.
        \item \verb'PARU_ORDERING_NONE':
            natural ordering.  The permutations $P$ and $Q$ are identity,
            unless singletons are removed prior to factorization.
        \end{itemize}

    \item \verb'PARU_CONTROL_RELAXED_AMALGAMATION': % goal # pivots in fronts
        threshold for relaxed amalgamation. When constructing its frontal
        matrices, ParU attempts to ensure that all frontal matrices contain at
        least this many pivot columns.  Values less than zero are treated as
        the default (32), and values greater than 512 are treated as 512.

    \item \verb'PARU_CONTROL_PANEL_WIDTH': % # of pivots in a panel
        Width of panel for dense factorization of
        each frontal matrix.

    \item \verb'PARU_CONTROL_DGEMM_TINY': % dimension of tiny dgemm's
        Do not call the BLAS \verb'dgemm' routine if all dimenions of its dense
        matrices are small than this threshold.

    \item \verb'PARU_CONTROL_DGEMM_TASKED': % dimension of tasked dgemm's
        When calling \verb'dgemm', if any dimension of its matriices are at or
        above this threshold, then a tasked variant of \verb'dgemm' is used.
        For the Intel MKL BLAS library, this is a standard call to
        \verb'dgemm', controlled by \verb'mkl_set_num_threads_local'.  For
        other BLAS library, ParU makes multiple calls to \verb'dgemm' using a
        single thread each.

    \item \verb'PARU_CONTROL_DTRSM_TASKED': % dimension of tasked dtrsm's
        When calling \verb'dtrsm', if any dimension of its matriices are at or
        above this threshold, then a tasked variant of \verb'dtrsm' is used.
        For the Intel MKL BLAS library, this is a standard call to
        \verb'dtrsm', controlled by \verb'mkl_set_num_threads_local'.  For
        other BLAS library, ParU makes multiple calls to \verb'dtrsm' using a
        single thread each.

    \item \verb'PARU_CONTROL_PRESCALE': % prescale input matrix, or not

        \begin{itemize}
        \item \verb'PARU_PRESCALE_MAX':  each row is scaled by
        the maximum absolute value in the row.
        \item \verb'PARU_PRESCALE_SUM':  each row is scaled by
        the sum of absolute values in the row.
        \item \verb'PARU_PRESCALE_NONE': no scaling is performed.
        \end{itemize}

    \item \verb'PARU_CONTROL_SINGLETONS': % filter singletons, or not
        If nonzero, singletons are permuted to the front of the matrix before
        factorization.  If zero, singletons are left as-is and not treated
        specially.

    \item \verb'PARU_CONTROL_MEM_CHUNK': % chunk size of memset and memcpy
        chunk size for parallel memset and memcpy.
    \end{itemize}

    For \verb'double' parameters:

    \begin{itemize}
    \item \verb'PARU_CONTROL_PIVOT_TOLERANCE': % pivot tolerance
        Pivot tolerance for off-diagonal pivots, or for all pivots when using
        the unsymmetric strategy.  A pivot is chosen if it is at least as large
        as 0.1 (default) times the maximum absolute value in its column.  This
        threshold allows for the selection of sparse pivot rows.  Standard
        partial pivoting is used with the tolerance is 1.0.

    \item \verb'PARU_CONTROL_DIAG_PIVOT_TOLERANCE': % diagonal pivot tolerance
        Pivot tolerance for diagonal pivots when using the symmetric strategy.

    \end{itemize}

Default values of Control parameters are defined below:

    {\footnotesize
    \begin{verbatim}
    #define PARU_DEFAULT_MAX_THREADS            (0)
    #define PARU_DEFAULT_STRATEGY               PARU_STRATEGY_AUTO
    #define PARU_DEFAULT_UMFPACK_STRATEGY       UMFPACK_STRATEGY_AUTO
    #define PARU_DEFAULT_ORDERING               PARU_ORDERING_METIS_GUARD
    #define PARU_DEFAULT_RELAXED_AMALGAMATION   (32)
    #define PARU_DEFAULT_PANEL_WIDTH            (32)
    #define PARU_DEFAULT_DGEMM_TINY             (4)
    #define PARU_DEFAULT_DGEMM_TASKED           (512)
    #define PARU_DEFAULT_DTRSM_TASKED           (4096)
    #define PARU_DEFAULT_PRESCALE               PARU_PRESCALE_MAX
    #define PARU_DEFAULT_SINGLETONS             (1)
    #define PARU_DEFAULT_MEM_CHUNK              (1024*1024)
    #define PARU_DEFAULT_PIVOT_TOLERANCE        (0.1)
    #define PARU_DEFAULT_DIAG_PIVOT_TOLERANCE   (0.001) \end{verbatim}}

Refer to the next section for how to use \verb'ParU_Get' to query
the current settings of parameters in the Control object.

%-------------------------------------------------------------------------------
\subsection{{\sf ParU\_Get}: get information from a ParU opaque object}
\label{get}
%-------------------------------------------------------------------------------

The \verb'ParU_Get' method returns properties fron any of the three opaque ParU
data structures: the \verb'ParU_Control' object, the \verb'ParU_Symbolic'
object containing a symbolic analysis, and the \verb'ParU_Numeric' object
containing a numeric factorization.

There are several signatures for \verb'ParU_Get' depending on which
object is being queried.  To query the \verb'ParU_Control' object,
the \verb'ParU_Control_enum' is used (see Section~\ref{set}).  To query the
other two objects (\verb'ParU_Symbolic' and  \verb'ParU_Numeric'), the
\verb'ParU_Get_enum' is used, described below.

    {\footnotesize
    \begin{verbatim}
    // enum for ParU_Get for Symbolic/Numeric objects
    typedef enum
    {
        // int64_t scalars:
        PARU_GET_N = 0,                 // # of rows/columns of A and its factors
        PARU_GET_ANZ = 1,               // # of entries in input matrix
        PARU_GET_LNZ_BOUND = 2,         // # of entries held in L
        PARU_GET_UNZ_BOUND = 3,         // # of entries held in U
        PARU_GET_NROW_SINGLETONS = 4,   // # of row singletons
        PARU_GET_NCOL_SINGLETONS = 5,   // # of column singletons
        PARU_GET_STRATEGY = 6,          // strategy used by ParU
        PARU_GET_UMFPACK_STRATEGY = 7,  // strategy used by UMFPACK
        PARU_GET_ORDERING = 8,          // ordering used by UMFPACK

        // int64_t arrays of size n:
        PARU_GET_P = 101,               // partial pivoting row ordering
        PARU_GET_Q = 102,               // fill-reducing column ordering

        // double scalars:
        PARU_GET_FLOPS_BOUND = 201,     // flop count for factorization (bound)
        PARU_GET_RCOND_ESTIMATE = 202,  // rcond estimate
        PARU_GET_MIN_UDIAG = 203,       // min (abs (diag (U)))
        PARU_GET_MAX_UDIAG = 204,       // max (abs (diag (U)))

        // double array of size n:
        PARU_GET_ROW_SCALE_FACTORS = 301,   // row scaling factors
    }
    ParU_Get_enum ; \end{verbatim}}

Use the following signatures to query the symbolic or numeric factorization:

    {\footnotesize
    \begin{verbatim}
    ParU_Info ParU_Get              // get int64_t from the symbolic/numeric objects
    (
        // input:
        const ParU_Symbolic Sym,    // symbolic analysis from ParU_Analyze
        const ParU_Numeric Num,     // numeric factorization from ParU_Factorize
        ParU_Get_enum field,        // field to get
        // output:
        int64_t *result,            // int64_t result: a scalar or an array
        // control:
        ParU_Control Control
    ) ;

    ParU_Info ParU_Get              // get double from the symbolic/numeric objects
    (
        // input:
        const ParU_Symbolic Sym,    // symbolic analysis from ParU_Analyze
        const ParU_Numeric Num,     // numeric factorization from ParU_Factorize
        ParU_Get_enum field,        // field to get
        // output:
        double *result,             // double result: a scalar or an array
        // control:
        ParU_Control Control
    ) ; \end{verbatim}}

These fields are described below.

\begin{itemize}

\item \verb'PARU_GET_N': the number of rows and columns of the
    matrices $A$, $L$, and $U$.

\item \verb'PARU_GET_ANZ': the number of entries in the input matrix $A$.

\item \verb'PARU_GET_LNZ_BOUND': the number of entries held in the data
    structure for $L$.  This is an upper bound on the number of nonzeros
    in $L$, because it includes extra zeros due to amalgamation.

\item \verb'PARU_GET_UNZ_BOUND': the number of entries held in the data
    structure for $U$.  This is an upper bound on the number of nonzeros
    in $U$, because it includes extra zeros due to amalgamation.

\item \verb'PARU_GET_NROW_SINGLETONS': number of row singletons, which
    are rows of $A$ with just a single entry (or become so when other
    singletons are removed).  These become diagonal pivot entries,
    where the corresponding row of $U$ has just a single entry.

\item \verb'PARU_GET_NCOL_SINGLETONS': number of column singletons, which
    are columns of $A$ with just a single entry (or become so when other
    singletons are removed).  These become diagonal pivot entries,
    where the corresponding column of $L$ has just a single entry.

\item \verb'PARU_GET_STRATEGY': the strategy selected by ParU, either
    \verb'PARU_STRATEGY_UNSYMMETRIC' or
    \verb'PARU_STRATEGY_SYMMETRIC'.

\item \verb'PARU_GET_UMFPACK_STRATEGY': the strategy selected by
    UMFPACK for the symbolic analysis phase of ParU, either
    \verb'UMFPACK_STRATEGY_UNSYMMETRIC' or \newline
    \verb'UMFPACK_STRATEGY_SYMMETRIC'.

\item \verb'PARU_GET_ORDERING':
    The ordering used during the symbolic analysis phase of ParU.
    See the list in Section~\ref{set} under the description of
    \verb'PARU_CONTROL_ORDERING'.

% // int64_t arrays of size n:
\item \verb'PARU_GET_P': partial pivoting row ordering,
    an \verb'int64_t' array of size \verb'n' where $A$ is
    \verb'n'-by-\verb'n'.

\item \verb'PARU_GET_Q': fill-reducing column ordering,
    an \verb'int64_t' array of size \verb'n' where $A$ is
    \verb'n'-by-\verb'n'.

% // double scalars:
\item \verb'PARU_GET_FLOPS_BOUND':
    an upper bound on the number of floating-operations
    performed to compute the LU factorization.  This includes
    extra flops due to amalgamation of frontal matrices.
    It does not include the prescaling of $A$, which takes
    an additional \verb'anz' flops (see \verb'PARU_GET_ANZ').

\item \verb'PARU_GET_RCOND_ESTIMATE': a rough estimate of the
    reciprical of the condition number of $A$, equal to the
    minimum absolute value on the diagonal of $U$,
    divided by the
    maximum absolute value on the diagonal of $U$.
\item \verb'PARU_GET_MIN_UDIAG':
    the minimum absolute value on the diagonal of $U$.
\item \verb'PARU_GET_MAX_UDIAG':
    the maximum absolute value on the diagonal of $U$.

% // double array of size n:
\item \verb'PARU_GET_ROW_SCALE_FACTORS':
    The row scaling factors, a \verb'double' array of size \verb'n'.
\end{itemize}


For example, to get a copy of the size-n column permutation vector from the
Symbolic object:

    {\footnotesize
    \begin{verbatim}
    int64_t Q [n] ;
    ParU_Get (Sym, Num, PARU_GET_Q, Q, Control) ; \end{verbatim}}

Most of the \verb'int64_t' results can be obtained with a NULL numeric object,
with the exception of the row permutation \verb'P', and the count of the number
of entries in the \verb'L' and \verb'U' factors.
All of the \verb'double' results require both the \verb'Sym' and \verb'Num'
objects to be valid.

The following three signatures are available for querying contents of the
Control object:

    {\footnotesize
    \begin{verbatim}
    ParU_Info ParU_Get              // get int64_t parameter from Control
    (
        // input
        ParU_Control_enum field,    // field to get
        // output:
        int64_t *c,                 // value of field
        // control:
        ParU_Control Control
    ) ;

    ParU_Info ParU_Get              // get double parameter from Control
    (
        // input
        ParU_Control_enum field,    // field to get
        // output:
        double *c,                  // value of field
        // control:
        ParU_Control Control
    ) ;

    ParU_Info ParU_Get              // get string from Control
    (
        // input:
        ParU_Control_enum field,    // field to get
        // output:
        const char **result,        // string result
        // control:
        ParU_Control Control
    ) ; \end{verbatim}}

The parameters that can be returned are the same as those described in
Section~\ref{set}, with three additional parameters that can be queried by
\verb'ParU_Get' but not set with \verb'ParU_Set'.  Two cases return string that
is owned by the library and must not be modified:

    {\footnotesize
    \begin{verbatim}
    char *blas_libary_name, *front_tasking ;
    ParU_Get (PARU_CONTROL_BLAS_LIBRARY_NAME, &blas_name, Control)) ;
    ParU_Get (PARU_CONTROL_FRONT_TREE_TASKING, &front_tasking, Control)) ; \end{verbatim}}

A final case returns true if ParU was compiled with OpenMP, or false otherwise:

    {\footnotesize
    \begin{verbatim}
    int64_t openmp_used ;
    ParU_Get (PARU_CONTROL_OPENMP , &openmp_used, Control)) ; \end{verbatim}}

%-------------------------------------------------------------------------------
\subsection{{\sf ParU\_Analyze}: symbolic analysis}
%-------------------------------------------------------------------------------

    {\footnotesize
    \begin{verbatim}
    ParU_Info ParU_Analyze
    (
        // input:
        cholmod_sparse *A,          // input matrix to analyze of size n-by-n
        // output:
        ParU_Symbolic *Sym_handle,  // output, symbolic analysis
        // control:
        ParU_Control Control
    ) ; \end{verbatim}}

    \verb'ParU_Analyze' takes as input a sparse matrix in the CHOLMOD data
    structure, \verb'A'.  The matrix must be square, double precision, not
    complex, and not held in the CHOLMOD symmetric storage format.  Refer to
    the CHOLMOD documentation for details.  On output, the symbolic analysis
    structure \verb'Sym' is created, passed in as \verb'&Sym'.  The symbolic
    analysis can be used for different calls to \verb'ParU_Factorize' for
    matrices that have the same sparsity pattern but different numerical
    values.  The symbolic analysis structure must be freed by
    \verb'ParU_FreeSymbolic'.

%-------------------------------------------------------------------------------
\subsection{{\sf ParU\_Factorize}: numerical factorization}
%-------------------------------------------------------------------------------

    {\footnotesize
    \begin{verbatim}
    ParU_Info ParU_Factorize
    (
        // input:
        cholmod_sparse *A,          // input matrix to factorize
        const ParU_Symbolic Sym,    // symbolic analysis from ParU_Analyze
        // output:
        ParU_Numeric *Num_handle,
        // control:
        ParU_Control Control
    ) ; \end{verbatim}}

    \verb'ParU_Factorize' performs the numerical factorization of its input
    sparse matrix \verb'A'.  The symbolic analsys \verb'Sym' must have been
    created by a prior call to \verb'ParU_Analyze' with the same matrix
    \verb'A', or one with the same sparsity pattern as the one passed to
    \verb'ParU_Factorize'.  On output, the \verb'&Num' structure is created.
    The numeric factorization structure must be freed by
    \verb'ParU_FreeNumeric'.

%-------------------------------------------------------------------------------
\subsection{{\sf ParU\_Solve}: solve a linear system, $Ax=b$}
%-------------------------------------------------------------------------------

    \verb'ParU_Solve' solves a sparse linear system $Ax=b$ for a sparse
    matrix \verb'A' and vectors \verb'x' and \verb'b', or matrices
    \verb'X' and \verb'B'.  The matrix \verb'A' must have been factorized
    by \verb'ParU_Factorize', and the \verb'Sym' and \verb'Num' structures
    from that call must be passed to this method.

    The method has four overloaded signatures, so that it can handle a single
    right-hand-side vector or a matrix with multiple right-hand-sides, and it
    provides the option of overwriting the input right-hand-side(s) with the
    solution(s).

    {\footnotesize
    \begin{verbatim}
    ParU_Info ParU_Solve            // solve Ax=b, overwriting b with solution x
    (
        // input:
        const ParU_Symbolic Sym,    // symbolic analysis from ParU_Analyze
        const ParU_Numeric Num,     // numeric factorization from ParU_Factorize
        // input/output:
        double *x,                  // vector of size n-by-1; right-hand on input,
                                    // solution on output
        // control:
        ParU_Control Control
    ) ; \end{verbatim} }

    {\footnotesize
    \begin{verbatim}
    ParU_Info ParU_Solve            // solve Ax=b
    (
        // input:
        const ParU_Symbolic Sym,    // symbolic analysis from ParU_Analyze
        const ParU_Numeric Num,     // numeric factorization from ParU_Factorize
        double *b,                  // vector of size n-by-1
        // output
        double *x,                  // vector of size n-by-1
        // control:
        ParU_Control Control
    ) ; \end{verbatim} }

    {\footnotesize
    \begin{verbatim}
    ParU_Info ParU_Solve            // solve AX=B, overwriting B with solution X
    (
        // input
        const ParU_Symbolic Sym,    // symbolic analysis from ParU_Analyze
        const ParU_Numeric Num,     // numeric factorization from ParU_Factorize
        int64_t nrhs,               // # of right-hand sides
        // input/output:
        double *X,                  // X is n-by-nrhs, where A is n-by-n;
                                    // holds B on input, solution X on input
        // control:
        ParU_Control Control
    ) ; \end{verbatim} }

    {\footnotesize
    \begin{verbatim}
    ParU_Info ParU_Solve            // solve AX=B
    (
        // input
        const ParU_Symbolic Sym,    // symbolic analysis from ParU_Analyze
        const ParU_Numeric Num,     // numeric factorization from ParU_Factorize
        int64_t nrhs,               // # of right-hand sides
        double *B,                  // n-by-nrhs, in column-major storage
        // output:
        double *X,                  // n-by-nrhs, in column-major storage
        // control:
        ParU_Control Control
    ) ; \end{verbatim} }

%-------------------------------------------------------------------------------
\subsection{{\sf ParU\_LSolve}: solve a linear system, $Lx=b$}
%-------------------------------------------------------------------------------

    \verb'ParU_LSolve' solves a lower triangular system, $Lx=b$ with vectors
    $x$ and $b$, or $LX=B$ with matrices $X$ and $B$, using the lower
    triangular factor computed by \verb'ParU_Factorize'.  No scaling or
    permutations are used.

    {\footnotesize
    \begin{verbatim}
    ParU_Info ParU_LSolve           // solve Lx=b
    (
        // input
        const ParU_Symbolic Sym,    // symbolic analysis from ParU_Analyze
        const ParU_Numeric Num,     // numeric factorization from ParU_Factorize
        // input/output:
        double *x,                  // n-by-1, in column-major storage;
                                    // holds b on input, solution x on input
        // control:
        ParU_Control Control
    ) ; \end{verbatim} }

    {\footnotesize
    \begin{verbatim}
    ParU_Info ParU_LSolve           // solve LX=B
    (
        // input
        const ParU_Symbolic Sym,    // symbolic analysis from ParU_Analyze
        const ParU_Numeric Num,     // numeric factorization from ParU_Factorize
        int64_t nrhs,               // # of right-hand-sides (# columns of X)
        // input/output:
        double *X,                  // X is n-by-nrhs, where A is n-by-n;
                                    // holds B on input, solution X on input
        // control:
        ParU_Control Control
    ) ; \end{verbatim} }

%-------------------------------------------------------------------------------
\subsection{{\sf ParU\_USolve}: solve a linear system, $Ux=b$}
%-------------------------------------------------------------------------------

    \verb'ParU_USolve' solves an upper triangular system, $Ux=b$ with vectors
    $x$ and $b$, or $UX=B$ with matrices $X$ and $B$, using the upper
    triangular factor computed by \verb'ParU_Factorize'.  No scaling or
    permutations are used.

    {\footnotesize
    \begin{verbatim}
    ParU_Info ParU_USolve           // solve Ux=b
    (
        // input
        const ParU_Symbolic Sym,    // symbolic analysis from ParU_Analyze
        const ParU_Numeric Num,     // numeric factorization from ParU_Factorize
        // input/output
        double *x,                  // n-by-1, in column-major storage;
                                    // holds b on input, solution x on input
        // control:
        ParU_Control Control
    ) ; \end{verbatim} }

    {\footnotesize
    \begin{verbatim}
    ParU_Info ParU_USolve           // solve UX=B
    (
        // input
        const ParU_Symbolic Sym,    // symbolic analysis from ParU_Analyze
        const ParU_Numeric Num,     // numeric factorization from ParU_Factorize
        int64_t nrhs,               // # of right-hand-sides (# columns of X)
        // input/output:
        double *X,                  // X is n-by-nrhs, where A is n-by-n;
                                    // holds B on input, solution X on input
        // control:
        ParU_Control Control
    ) ; \end{verbatim} }

%-------------------------------------------------------------------------------
\subsection{{\sf ParU\_Perm}: permute and scale a dense vector or matrix}
\label{perm}
%-------------------------------------------------------------------------------

    \verb'ParU_Perm' permutes and optionally scales a vector $b$ or matrix $B$.
    If the input \verb's' is \verb'NULL', no scaling is applied.  The
    permutation vector \verb'P' has size \verb'n'.  If the $k$th index in the
    permutation is row $i$, then \verb'i = P[k]'.

    For the vector case, the output is $x(k) = b(P(k)) / s(P(k))$, or
    $x(k) = b(P(k))$, or if \verb's' is \verb'NULL', for all $k$ in the range 0
    to $n-1$.

    For the matrix case, the output is $X(k,j) = B(P(k),j) / s(P(k))$ for all
    rows $k$ and all columns $j$ of $X$ and $B$.  If \verb's' is \verb'NULL',
    then the output is $X(k,j) = B(P(k),j)$.

    {\footnotesize
    \begin{verbatim}
    ParU_Info ParU_Perm
    (
        // inputs
        const int64_t *P,   // permutation vector of size n
        const double *s,    // vector of size n (optional)
        const double *b,    // vector of size n
        int64_t n,          // length of P, s, B, and X
        // output
        double *x,          // vector of size n
        // control:
        ParU_Control Control
    ) ; \end{verbatim} }

    {\footnotesize
    \begin{verbatim}
    ParU_Info ParU_Perm
    (
        // inputs
        const int64_t *P,   // permutation vector of size nrows
        const double *s,    // vector of size nrows (optional)
        const double *B,    // array of size nrows-by-ncols
        int64_t nrows,      // # of rows of X and B
        int64_t ncols,      // # of columns of X and B
        // output
        double *X,          // array of size nrows-by-ncols
        // control:
        ParU_Control Control
    ) ; \end{verbatim} }

%-------------------------------------------------------------------------------
\subsection{{\sf ParU\_InvPerm}: permute and scale a dense vector or matrix}
\label{invperm}
%-------------------------------------------------------------------------------

    \verb'ParU_InvPerm' permutes and optionally scales a vector $b$ or matrix
    $B$.  If the input \verb's' is \verb'NULL', no scaling is applied.  The
    permutation vector \verb'P' has size \verb'n', and its inverse is
    implicitly used by this method.  If the $k$th index in the permutation is
    row $i$, then \verb'i = P[k]'.

    For the vector case, the output is $x(P(k)) = b(k) / s(P(k))$, or
    $x(P(k)) = b(k)$, or if \verb's' is \verb'NULL', for all $k$ in the range 0
    to $n-1$.

    For the matrix case, the output is $X(P(k),j) = B(k,j) / s(P(k))$ for all
    rows $k$ and all columns $j$ of $X$ and $B$.  If \verb's' is \verb'NULL',
    then the output is $X(P(k),j) = B(k,j)$.

    {\footnotesize
    \begin{verbatim}
    ParU_Info ParU_InvPerm
    (
        // inputs
        const int64_t *P,   // permutation vector of size n
        const double *s,    // vector of size n (optional)
        const double *b,    // vector of size n
        int64_t n,          // length of P, s, B, and X
        // output
        double *x,          // vector of size n
        // control:
        ParU_Control Control
    ) ; \end{verbatim} }

    {\footnotesize
    \begin{verbatim}
    ParU_Info ParU_InvPerm
    (
        // inputs
        const int64_t *P,   // permutation vector of size nrows
        const double *s,    // vector of size nrows (optional)
        const double *B,    // array of size nrows-by-ncols
        int64_t nrows,      // # of rows of X and B
        int64_t ncols,      // # of columns of X and B
        // output
        double *X,          // array of size nrows-by-ncols
        // control:
        ParU_Control Control
    ) ; \end{verbatim} }

    The \verb'ParU_LSolve', \verb'ParU_USolve', \verb'ParU_Perm', and
    \verb'ParU_InvPerm' can be used together to solve $Ax=b$ or $AX=B$.  For
    example, if \verb't' is a temporary vector of size \verb'n', and $A$ is an
    $n$-by-$n$ matrix, calling \verb'ParU_Solve' to solve $Ax=b$ is identical
    to the following (ignoring any tests for error conditions):

    {\footnotesize
    \begin{verbatim}
    int64_t P [n], Q [n] ;
    double t [n], R [n] ;
    ParU_Get (Sym, Num, PARU_GET_P, P, Control) ;
    ParU_Get (Sym, Num, PARU_GET_Q, Q, Control) ;
    ParU_Get (Sym, Num, PARU_GET_ROW_SCALE_FACTORS, R, Control) ;
    ParU_Perm (P, Rs, b, n, t, Control) ;
    ParU_LSolve (Sym, Num, t, Control) ;
    ParU_USolve (Sym, Num, t, Control) ;
    ParU_InvPerm (Q, NULL, t, n, x, Control) ; \end{verbatim} }

    The numeric factorization \verb'Num' contains the row permutation vector
    \verb'P' from partial pivoting, and the row scaling vector
    \verb'R'.  The symbolic analysis structure \verb'Sym' contains the
    fill-reducing column preordering, \verb'Q'.

%-------------------------------------------------------------------------------
\subsection{{\sf ParU\_Residual}: compute the residual}
%-------------------------------------------------------------------------------

    The \verb'ParU_Residual' function computes the relative residual of
    $Ax=b$ or $AX=B$, in the 1-norm.  It also computes the 1-norm of $A$
    and the solution $X$ or $x$.

    {\footnotesize
    \begin{verbatim}
    ParU_Info ParU_Residual
    (
        // inputs:
        cholmod_sparse *A,  // an n-by-n sparse matrix
        double *x,          // vector of size n, solution to Ax=b
        double *b,          // vector of size n
        // output:
        double &resid,      // residual: norm1(b-A*x) / (norm1(A) * norm1 (x))
        double &anorm,      // 1-norm of A
        double &xnorm,      // 1-norm of x
        // control:
        ParU_Control Control
    ) ; \end{verbatim} }

    {\footnotesize
    \begin{verbatim}
    ParU_Info ParU_Residual
    (
        // inputs:
        cholmod_sparse *A,  // an n-by-n sparse matrix
        double *X,          // array of size n-by-nrhs, solution to AX=B
        double *B,          // array of size n-by-nrhs
        int64_t nrhs,
        // output:
        double &resid,      // residual: norm1(B-A*X) / (norm1(A) * norm1 (X))
        double &anorm,      // 1-norm of A
        double &xnorm,      // 1-norm of X
        // control:
        ParU_Control Control
    ) ; \end{verbatim} }

%-------------------------------------------------------------------------------
\subsection{{\sf ParU\_FreeNumeric}: free a numeric factorization}
%-------------------------------------------------------------------------------

    {\footnotesize
    \begin{verbatim}
    ParU_Info ParU_FreeNumeric
    (
        // input/output:
        ParU_Numeric *Num_handle,       // numeric object to free
        // control:
        ParU_Control Control
    ) ; \end{verbatim} }

%-------------------------------------------------------------------------------
\subsection{{\sf ParU\_FreeSymbolic}: free a symbolic analysis}
%-------------------------------------------------------------------------------

    {\footnotesize
    \begin{verbatim}
    ParU_Info ParU_FreeSymbolic
    (
        // input/output:
        ParU_Symbolic *Sym_handle,      // symbolic object to free
        // control:
        ParU_Control Control
    ) ; \end{verbatim} }

%-------------------------------------------------------------------------------
\subsection{{\sf ParU\_FreeControl}: free a Control object}
%-------------------------------------------------------------------------------

    {\footnotesize
    \begin{verbatim}
    ParU_Info ParU_FreeControl
    (
        // input/output:
        ParU_Control *Control_handle    // Control object to free
    ) ; \end{verbatim} }

%-------------------------------------------------------------------------------
\section{C Syntax}
%-------------------------------------------------------------------------------

The C interface is quite similar to the C++ interface.  The next sections
describe the user-callable C functions, their prototypes, and what they can do.

%-------------------------------------------------------------------------------
\subsection{{\sf ParU\_C\_Version}: version of the ParU package}
%-------------------------------------------------------------------------------

    {\footnotesize
    \begin{verbatim}
    ParU_Info ParU_C_Version (int ver [3], char date [128]) ; \end{verbatim} }

%-------------------------------------------------------------------------------
\subsection{{\sf ParU\_C\_Control}: parameters that control ParU}
%-------------------------------------------------------------------------------

    Any C ParU function can be passed a NULL pointer for its \verb'Control'
    parameter.  In that case, defaults are used.  To use non-default
    parameters, create a \verb'Control' object and then set its parameters.
    The object is freed by \verb'ParU_C_FreeControl'.

    The \verb'ParU_C_Control' structure contains parameters that control
    various ParU options.  The object is created by \verb'ParU_C_InitControl',
    modified by \verb'ParU_C_Set_Control_*', and deleted by
    \verb'ParU_C_FreeControl'.  Its contents can be queried with
    \verb'ParU_C_Get_Control_INT64'.  and \verb'ParU_C_Get_Control_FP64'.

    Any ParU function can be passed a NULL pointer for its \verb'Control'
    parameter.  In that case, defaults are used.  To use non-default
    parameters, create a \verb'Control' object and then set its parameters.

    {\footnotesize
    \begin{verbatim}
    ParU_Info ParU_C_InitControl
    (
        ParU_C_Control *Control_C_handle    // Control object to create
    ) ; \end{verbatim} }

    {\footnotesize
    \begin{verbatim}
    ParU_Info ParU_C_FreeControl
    (
        ParU_C_Control *Control_handle_C    // Control object to free
    ) ; \end{verbatim} }

%-------------------------------------------------------------------------------
\subsection{{\sf ParU\_C\_Get\_*}: get information from a ParU opaque object}
%-------------------------------------------------------------------------------

The \verb'ParU_C_Get_*' methods retrieve scalars or arrays from the C versions
of the Control, Symbolic, or Numeric objects.  See Section~\ref{get} for
details.

    {\footnotesize
    \begin{verbatim}

    ParU_Info ParU_C_Get_INT64       // get int64_t contents of Sym_C and Num_C
    (
        // input:
        const ParU_C_Symbolic Sym_C, // symbolic analysis from ParU_C_Analyze
        const ParU_C_Numeric Num_C,  // numeric factorization from ParU_C_Factorize
        ParU_Get_enum field,         // field to get
        // output:
        int64_t *result,             // int64_t result: a scalar or an array
        // control:
        ParU_C_Control Control_C
    ) ;

    ParU_Info ParU_C_Get_FP64        // get double contents of Sym_C and Num_C
    (
        // input:
        const ParU_C_Symbolic Sym_C, // symbolic analysis from ParU_C_Analyze
        const ParU_C_Numeric Num_C,  // numeric factorization from ParU_C_Factorize
        ParU_Get_enum field,         // field to get
        // output:
        double *result,              // double result: a scalar or an array
        // control:
        ParU_C_Control Control_C
    ) ;

    ParU_Info ParU_C_Get_Control_INT64  // get int64_t contents of Control
    (
        // input:
        ParU_Control_enum field,      // field to get
        // output:
        int64_t *result,              // int64_t result: a scalar or an array
        // control:
        ParU_C_Control Control_C
    ) ;

    ParU_Info ParU_C_Get_Control_FP64   // get double contents of Control
    (
        // input:
        ParU_Control_enum field,      // field to get
        // output:
        double *result,               // int64_t result: a scalar or an array
        // control:
        ParU_C_Control Control_C
    ) ;

    ParU_Info ParU_C_Get_Control_CONSTCHAR   // get string from Control
    (
        // input:
        ParU_Control_enum field,      // field to get
        // output:
        const char **result,          // string result
        // control:
        ParU_C_Control Control_C
    ) ; \end{verbatim} }


%-------------------------------------------------------------------------------
\subsection{{\sf ParU\_C\_Set\_Control\_*}: set Control parameters}
%-------------------------------------------------------------------------------

The \verb'ParU_C_Set_Control_*' methods set parameters in the C version of the
Control object.  See Section~\ref{control} for details.

    {\footnotesize
    \begin{verbatim}

    ParU_Info ParU_C_Set_Control_INT64      // set int64_t parameter in Control
    (
        // input
        ParU_Control_enum field,    // field to set
        int64_t c,                  // value to set it to
        // control:
        ParU_C_Control Control_C
    ) ;

    ParU_Info ParU_C_Set_Control_FP64       // set double parameter in Control
    (
        // input
        ParU_Control_enum field,    // field to set
        double c,                   // value to set it to
        // control:
        ParU_C_Control Control_C
    ) ; \end{verbatim} }

%-------------------------------------------------------------------------------
\subsection{{\sf ParU\_C\_Analyze}: symbolic analysis}
%-------------------------------------------------------------------------------

    \verb'ParU_C_Analyze' performs the symbolic analysis of a sparse
        matrix, based solely on its nonzero pattern.  \verb'ParU_C_Analyze' is
        called once and can be used for different \verb'ParU_C_Factorize' calls
        for the matrices that have the same pattern but different numerical
        values.  The symbolic analysis structure must be freed by
        \verb'ParU_C_FreeSymbolic'.

    {\footnotesize
    \begin{verbatim}
    ParU_Info ParU_C_Analyze
    (
        // input:
        cholmod_sparse *A,  // input matrix to analyze of size n-by-n
        // output:
        ParU_C_Symbolic *Sym_handle_C,  // output, symbolic analysis
        // control:
        ParU_C_Control Control_C
    ) ; \end{verbatim} }

%-------------------------------------------------------------------------------
\subsection{{\sf ParU\_C\_Factorize}: numeric factorization}
%-------------------------------------------------------------------------------

    \verb'ParU_C_Factorize' computes the numeric factorization.  The
    \verb'ParU_C_Symbolic' structure computed in \verb'ParU_C_Analyze' is an
    input to this routine.  The numeric factorization structure must be freed
    by \verb'ParU_C_FreeNumeric'.

    {\footnotesize
    \begin{verbatim}
    ParU_Info ParU_C_Factorize
    (
        // input:
        cholmod_sparse *A,              // input matrix to factorize of size n-by-n
        const ParU_C_Symbolic Sym_C,    // symbolic analysis from ParU_Analyze
        // output:
        ParU_C_Numeric *Num_handle_C,   // output numerical factorization
        // control:
        ParU_C_Control Control_C
    ) ; \end{verbatim} }

%-------------------------------------------------------------------------------
\subsection{{\sf ParU\_C\_Solve\_A*}: solve a linear system, $Ax=b$}
%-------------------------------------------------------------------------------

    The \verb'ParU_C_Solve_Axx',  \verb'ParU_C_Solve_Axb',
    \verb'ParU_C_Solve_AXX' and \verb'ParU_C_Solve_AXB' methods solve a sparse
    linear system $Ax=b$ for a sparse matrix \verb'A' and vectors \verb'x' and
    \verb'b', or matrices \verb'X' and \verb'B'.  The matrix \verb'A' must have
    been factorized by \verb'ParU_Factorize', and the \verb'Sym' and \verb'Num'
    structures from that call must be passed to this method.

    {\footnotesize
    \begin{verbatim}
    ParU_Info ParU_C_Solve_Axx       // solve Ax=b, overwriting b with solution x
    (
        // input:
        const ParU_C_Symbolic Sym_C, // symbolic analysis from ParU_C_Analyze
        const ParU_C_Numeric Num_C,  // numeric factorization from ParU_C_Factorize
        // input/output:
        double *x,              // vector of size n-by-1; right-hand on input,
                                // solution on output
        // control:
        ParU_C_Control Control_C
    ) ; \end{verbatim} }

    {\footnotesize
    \begin{verbatim}
    ParU_Info ParU_C_Solve_Axb        // solve Ax=b
    (
        // input:
        const ParU_C_Symbolic Sym_C, // symbolic analysis from ParU_C_Analyze
        const ParU_C_Numeric Num_C,  // numeric factorization from ParU_C_Factorize
        double *b,              // vector of size n-by-1
        // output
        double *x,              // vector of size n-by-1
        // control:
        ParU_C_Control Control_C
    ) ; \end{verbatim} }

    {\footnotesize
    \begin{verbatim}
    ParU_Info ParU_C_Solve_AXX       // solve AX=B, overwriting B with solution X
    (
        // input
        const ParU_C_Symbolic Sym_C, // symbolic analysis from ParU_C_Analyze
        const ParU_C_Numeric Num_C,  // numeric factorization from ParU_C_Factorize
        int64_t nrhs,
        // input/output:
        double *X,              // array of size n-by-nrhs in column-major storage,
                                // right-hand-side on input, solution on output.
        // control:
        ParU_C_Control Control_C
    ) ; \end{verbatim} }

    {\footnotesize
    \begin{verbatim}
    ParU_Info ParU_C_Solve_AXB       // solve AX=B, overwriting B with solution X
    (
        // input
        const ParU_C_Symbolic Sym_C, // symbolic analysis from ParU_C_Analyze
        const ParU_C_Numeric Num_C,  // numeric factorization from ParU_C_Factorize
        int64_t nrhs,
        double *B,              // array of size n-by-nrhs in column-major storage
        // output:
        double *X,              // array of size n-by-nrhs in column-major storage
        // control:
        ParU_C_Control Control_C
    ) ; \end{verbatim} }

%-------------------------------------------------------------------------------
\subsection{{\sf ParU\_C\_Solve\_L*}: solve a linear system, $Lx=b$}
%-------------------------------------------------------------------------------

    The \verb'ParU_C_Solve_Lxx' and \verb'ParU_C_Solve_LXX' methods solve lower
    triangular systems, $Lx=b$ with vectors $x$ and $b$, or $LX=B$ with
    matrices $X$ and $B$, using the lower triangular factor computed by
    \verb'ParU_Factorize'.  No scaling or permutations are used.

    {\footnotesize
    \begin{verbatim}
    ParU_Info ParU_C_Solve_Lxx       // solve Lx=b, overwriting b with solution x
    (
        // input:
        const ParU_C_Symbolic Sym_C, // symbolic analysis from ParU_C_Analyze
        const ParU_C_Numeric Num_C,  // numeric factorization from ParU_C_Factorize
        // input/output:
        double *x,              // vector of size n-by-1; right-hand on input,
                                // solution on output
        // control:
        ParU_C_Control Control_C
    ) ; \end{verbatim} }

    {\footnotesize
    \begin{verbatim}
    ParU_Info ParU_C_Solve_LXX       // solve LX=B, overwriting B with solution X
    (
        // input
        const ParU_C_Symbolic Sym_C, // symbolic analysis from ParU_C_Analyze
        const ParU_C_Numeric Num_C,  // numeric factorization from ParU_C_Factorize
        int64_t nrhs,
        // input/output:
        double *X,              // array of size n-by-nrhs in column-major storage,
                                // right-hand-side on input, solution on output.
        // control:
        ParU_C_Control Control_C
    ) ; \end{verbatim} }

%-------------------------------------------------------------------------------
\subsection{{\sf ParU\_C\_Solve\_U*}: solve a linear system, $Ux=b$}
%-------------------------------------------------------------------------------

    The \verb'ParU_C_Solve_Uxx' and \verb'ParU_C_Solve_UXX' methods solve an
    upper triangular system, $Ux=b$ or $UX=B$.  No scaling or permutation is
    performed.

    {\footnotesize
    \begin{verbatim}
    ParU_Info ParU_C_Solve_Uxx       // solve Ux=b, overwriting b with solution x
    (
        // input:
        const ParU_C_Symbolic Sym_C, // symbolic analysis from ParU_C_Analyze
        const ParU_C_Numeric Num_C,  // numeric factorization from ParU_C_Factorize
        // input/output:
        double *x,              // vector of size n-by-1; right-hand on input,
                                // solution on output
        // control:
        ParU_C_Control Control_C
    ) ; \end{verbatim} }

    {\footnotesize
    \begin{verbatim}
    ParU_Info ParU_C_Solve_UXX       // solve UX=B, overwriting B with solution X
    (
        // input
        const ParU_C_Symbolic Sym_C, // symbolic analysis from ParU_C_Analyze
        const ParU_C_Numeric Num_C,  // numeric factorization from ParU_C_Factorize
        int64_t nrhs,
        // input/output:
        double *X,              // array of size n-by-nrhs in column-major storage,
                                // right-hand-side on input, solution on output.
        // control:
        ParU_C_Control Control_C
    ) ; \end{verbatim} }

%-------------------------------------------------------------------------------
\subsection{{\sf ParU\_C\_Perm}: permute and scale a dense vector or matrix}
%-------------------------------------------------------------------------------

    \verb'ParU_C_Perm' and \verb'ParU_C_Perm_X' permutes and optionally scale a
    dense vector or matrix.  Refer to Section \ref{perm} for details.

    {\footnotesize
    \begin{verbatim}
    ParU_Info ParU_C_Perm
    (
        // inputs
        const int64_t *P,   // permutation vector of size n
        const double *s,    // vector of size n (optional)
        const double *b,    // vector of size n
        int64_t n,          // length of P, s, B, and X
        // output
        double *x,          // vector of size n
        // control:
        ParU_C_Control Control_C
    ) ; \end{verbatim} }

    {\footnotesize
    \begin{verbatim}
    ParU_Info ParU_C_Perm_X
    (
        // inputs
        const int64_t *P,   // permutation vector of size nrows
        const double *s,    // vector of size nrows (optional)
        const double *B,    // array of size nrows-by-ncols
        int64_t nrows,      // # of rows of X and B
        int64_t ncols,      // # of columns of X and B
        // output
        double *X,          // array of size nrows-by-ncols
        // control:
        ParU_C_Control Control_C
    ) ; \end{verbatim} }

%-------------------------------------------------------------------------------
\subsection{{\sf ParU\_C\_InvPerm}: permute and scale a dense vector or matrix}
%-------------------------------------------------------------------------------

    \verb'ParU_C_InvPerm' and \verb'ParU_C_InvPerm_X' and permutes and
    optionally scale a dense vector or matrix.  Refer to Section \ref{invperm}
    for details.

    {\footnotesize
    \begin{verbatim}
    ParU_Info ParU_C_InvPerm
    (
        // inputs
        const int64_t *P,   // permutation vector of size n
        const double *s,    // vector of size n (optional)
        const double *b,    // vector of size n
        int64_t n,          // length of P, s, B, and X
        // output
        double *x,          // vector of size n
        // control
        ParU_C_Control Control_C
    ) ; \end{verbatim} }

    {\footnotesize
    \begin{verbatim}
    ParU_Info ParU_C_InvPerm_X
    (
        // inputs
        const int64_t *P,   // permutation vector of size nrows
        const double *s,    // vector of size nrows (optional)
        const double *B,    // array of size nrows-by-ncols
        int64_t nrows,      // # of rows of X and B
        int64_t ncols,      // # of columns of X and B
        // output
        double *X,          // array of size nrows-by-ncols
        // control
        ParU_C_Control Control_C
    ) ; \end{verbatim} }

%-------------------------------------------------------------------------------
\subsection{{\sf ParU\_C\_Residual\_*}: compute the residual}
%-------------------------------------------------------------------------------

    \verb'ParU_C_Residual_bAx' and \verb'ParU_C_Residual_BAX' compute the
    relative residual of $Ax=b$ or $AX=B$, in the 1-norm, and the 1-norm of $A$
    and the solution $X$ or $x$.

    {\footnotesize
    \begin{verbatim}
    ParU_Info ParU_C_Residual_bAx
    (
        // inputs:
        cholmod_sparse *A,  // an n-by-n sparse matrix
        double *x,          // vector of size n
        double *b,          // vector of size n
        // output:
        double *residc,     // residual: norm1(b-A*x) / (norm1(A) * norm1 (x))
        double *anormc,     // 1-norm of A
        double *xnormc,     // 1-norm of x
        // control:
        ParU_C_Control Control_C
    ) ; \end{verbatim} }

    {\footnotesize
    \begin{verbatim}
    ParU_Info ParU_C_Residual_BAX
    (
        // inputs:
        cholmod_sparse *A,  // an n-by-n sparse matrix
        double *X,          // array of size n-by-nrhs
        double *B,          // array of size n-by-nrhs
        int64_t nrhs,
        // output:
        double *residc,     // residual: norm1(B-A*X) / (norm1(A) * norm1 (X))
        double *anormc,     // 1-norm of A
        double *xnormc,     // 1-norm of X
        // control:
        ParU_C_Control Control_C
    ) ; \end{verbatim} }

%-------------------------------------------------------------------------------
\subsection{{\sf ParU\_C\_FreeNumeric}: free a numeric factorization}
%-------------------------------------------------------------------------------

    {\footnotesize
    \begin{verbatim}
    ParU_Info ParU_C_FreeNumeric
    (
        ParU_C_Numeric *Num_handle_C,    // numeric object to free
        // control:
        ParU_C_Control *Control_C
    ) ; \end{verbatim} }

%-------------------------------------------------------------------------------
\subsection{{\sf ParU\_C\_FreeSymbolic}: free a symbolic analysis structure}
%-------------------------------------------------------------------------------

    {\footnotesize
    \begin{verbatim}
    ParU_Info ParU_C_FreeSymbolic
    (
        ParU_C_Symbolic *Sym_handle_C,   // symbolic object to free
        // control:
        ParU_C_Control *Control_C
    ) ; \end{verbatim} }

%-------------------------------------------------------------------------------
\subsection{{\sf ParU\_C\_FreeControl}: free a Control object}
%-------------------------------------------------------------------------------

    {\footnotesize
    \begin{verbatim}
    ParU_Info ParU_C_FreeControl
    (
        ParU_C_Control *Control_handle_C    // Control object to free
    ) ; \end{verbatim} }

%-------------------------------------------------------------------------------
\section{Thread safety of {\sf malloc}, {\sf calloc}, {\sf realloc}, and
{\sf free}}
%-------------------------------------------------------------------------------

ParU is a C++ library but uses the C memory manager for all of its memory
allocations, for compatibility with the other packages in SuiteSparse.  It
makes limited use of the C++ \verb'new' and \verb'delete', but overrides those
functions to use \verb'SuiteSparse_malloc' and \verb'SuiteSparse_free'.  ParU
relies on the memory manager routines defined by the \newline
\verb'SuiteSparse_config'
library (\verb'SuiteSparse_malloc', \verb'SuiteSparse_calloc', \newline
\verb'SuiteSparse_realloc', and \verb'SuiteSparse_free').  By default, those
routines relies on the C \verb'malloc', \verb'calloc', \verb'realloc', and
\verb'free' methods, respectively.  They can be redefined; refer to the
documentation of \verb'SuiteSparse_config' on how to do this.

The \verb'malloc', \verb'calloc', \verb'realloc', and \verb'free' methods must
be thread-safe, since ParU calls those methods from within its parallel tasks.
All of their implementations in the standard C libraries that we are aware of
are thread-safe.  However, if your memory manager routines are not thread-safe,
ParU will fail catastrophically.

%-------------------------------------------------------------------------------
\section{Using ParU in MATLAB}
%-------------------------------------------------------------------------------

%-------------------------------------------------------------------------------
\subsection{Compiling ParU for MATLAB}
%-------------------------------------------------------------------------------

To use ParU in MATLAB, you must first compile the \verb'paru' mexFunction.
In MATLAB, go to the \verb'ParU/MATLAB' directory and type \verb'paru_make'.
Then add the \verb'ParU/MATLAB' directory to your MATLAB path for future use.

For best performance, paru relies on functions unique to the Intel MKL BLAS.
An optional input, \verb'paru_make(try_intel)', is true by default.
\verb'paru_make' detects the BLAS library used by MATLAB and then attempts to
use functions unique to the Intel MKL BLAS library
(\verb'mkl*set_num_threads_local').  This may fail when \verb'paru' is
compiled, in which case compilation is reattempted with \verb'try_intel'
false).  If \verb'paru' fails when it runs, with a link error reporting that an
an \verb'mkl_*' routine is not found, use \verb'paru_make(false)' to disable
the Intel MKL functions.

{\bf Limitations:}
The built-in compiler used by the MATLAB \verb'mex' command on Windoes does not
support OpenMP, so only parallelism within the BLAS can be used on Windows.

%-------------------------------------------------------------------------------
\subsection{Using ParU in MATLAB}
%-------------------------------------------------------------------------------

The basic usage \verb'x=paru(A,b)' solves the linear system $Ax=b$, computing
\verb'x=A\b'.  The matrix \verb'A' must be sparse, square, non-singular, and
real.

Additional options are available to change \verb'paru''s behavior, and
an additonal output parameter reports statistics on the algorithm:

    \begin{verbatim}
    \verb'[x,stats] = paru (A,b,opts)'
    \end{verbatim}

\verb'opts' is a struct containing the following fields.  Any field that is not
recognized is ignored, and missing fields are treated as their defaults:

\begin{itemize}

    \item \verb'opts.strategy':  ordering strategy,
        as a string (default: \verb"'auto'"):

        \begin{itemize}
        \item \verb"'auto'": the strategy is selected automatically.
        \item \verb"'symmetric'": ordering of \verb"A+A'", with preference for
            diagonal pivoting.  Works well for matrices with mostly symmetric
            nonzero pattern.
        \item \verb"'unsymmetric'": ordering \verb"A'*A", with no preference
            for diagonal pivoting.  Works well for matrices with unsymmetric
            nonzero pattern.
        \end{itemize}

    \item \verb'opts.tol':   relative pivot tolerance for off-diagonal entries
    (default: 0.1).  Pivot entries must be 0.1 times the max absolute value in
    its column.

    \item \verb'opts.diagtol':   relative pivot tolerance for diagonal pivot
    entries when using the symmetric strategy (default: 0.001).  A lower
    tolerance for diagonal entries tends to reduce fill-in.

    \item \verb'opts.ordering':  fill-reducing ordering option, as a string
    (default: \verb"'amd'"):

        \begin{itemize}
        \item \verb"'amd'": AMD for the symmetric strategy, COLAMD for
        unsymmetric.

        \item \verb"'cholmod'": use CHOLMOD's ordering strategy: try AMD or
        COLAMD, and then try METIS if the fill-in from AMD/COLAMD is high; then
        selects the best ordering found.

        \item \verb"'metis'": METIS on \verb"A+A'" for symmetric strategy,
            \verb"A'*A" for the unsymmetric strategy.

        \item \verb"'metis_guard'": use the \verb"'metis'" ordering unless the
        matrix has one or more rows with $3.2\sqrt{n}$ entries, in which case
        use \verb"'amd'".

        \item \verb"'none'": no fill-reducing ordering.

        \end{itemize}

    \item \verb'opts.prescale': prescaling the input matrix
        (default \verb"'max'").

        \begin{itemize}
        \item \verb"'none'": no prescaling.
        \item \verb"'sum'": scale each row by the sum of the absolute values
            of the row.  The prescaled matrix is $RA$ where
            \verb'R(i,i) = 1/sum(abs(A(i,:)))'.
        \item \verb"'max'": scale each row by the sum of the absolute values
            of the row.  The prescaled matrix is $RA$ where
            \verb'R(i,i) = 1/max(abs(A(i,:)))'.
        \end{itemize}

\end{itemize}

\verb'stats' is an optional output that provides information on the ParU
analysis and factorization of the matrix:

\begin{itemize}
    \item \verb'stats.analysis_time': symbolic analysis time in seconds.
    \item \verb'stats.factorization_time': numeric factorization time in
        seconds.
    \item \verb'stats.solve_time': forward/backward solve time in seconds.
    \item \verb'stats.strategy_used': symmetric or unsymmetric.
    \item \verb'stats.ordering_used':
        \verb"amd(A+A')", \verb"colamd(A)", \verb"metis(A+A')",
        \verb"metis(A'*A)", or none.
    \item \verb'stats.flops': flop count for LU factorization.
    \item \verb'stats.lnz': number of entries in L.
    \item \verb'stats.unz': number of entries in U.
    \item \verb'stats.rcond': rough estimate of the recripical of the condition
        number.
    \item \verb'stats.blas': BLAS library used, as a string.
    \item \verb'stats.front_tree_tasking': a string stating how the paru
    mexFunction was compiled, whether or not tasking is available for
    factorizing multiple fronts at the same time (\verb"'sequential'" or
    \verb"'parallel'").  Parallel tasking is required for best performance, and
    requires OpenMP tasking, which is available in OpenMP 4.0 and later.
\end{itemize}

%-------------------------------------------------------------------------------
\section{Requirements and Availability}
\label{summary}
%-------------------------------------------------------------------------------

ParU requires several Collected Algorithms of the ACM:
CHOLMOD \cite{ChenDavisHagerRajamanickam09,DavisHager09},
AMD \cite{AmestoyDavisDuff96,AmestoyDavisDuff03},
COLAMD \cite{DavisGilbertLarimoreNg00_algo,DavisGilbertLarimoreNg00}
and UMFPACK \cite{10.1145/992200.992206} for its ordering/analysis phase and
for its basic sparse matrix data structure, and the BLAS \cite{dddh:90} for
dense matrix computations on its frontal matrices.  An efficient implementation
of the BLAS is strongly recommended, such as the Intel MKL, the AMD ACML,
OpenBLAS, FLAME \cite{GotoVanDeGeijn08}. or vendor-provide BLAS. Note that
while ParU uses nested parallelism heavily the right options for the BLAS
library must be chosen to get a good performance.

For best performance, the Intel MKL BLAS is required.  ParU relies on the
\newline \verb'mkl_set_num_threads_local' function to select the number of
threads to use for the BLAS calls from each of its own threads, which required
nested parallelism.

ParU also relies heavily on OpenMP tasking to factorize multiple frontal
matrices at the same time, where each frontal matrix can also be factorized by
multiple threads.  If tasking is not available, each frontal matrix is
factorized one at a time (but still in parallel).  For best performance, nested
parallelism is required.  However, when using the \verb'gcc' compiler on
Windows and Mac, we have found that the OpenMP library can hang.  As a result,
on those platforms, nested parallelism is disabled when using \verb'gcc'.
If using \verb'gcc', use a recent compiler (version 7.5.0 fails; 12.2.0
works).

You can query ParU at run time to determine which BLAS library it is using, and
whether or not it is compiled to use parallel or sequential factorization of
its frontal matrix tree.  See \verb'ParU_Get' (Section~\ref{get}) for details.

SuiteSparse uses a slightly modified version of METIS 5.1.0, distributed along
with SuiteSparse itself.  Its use is optional, however. ParU uses AMD as its
default ordering. METIS tends to give orderings that are good for parallelism.
However, METIS itself can be slower than AMD. As a result, the symbolic
analysis using METIS can be slow, but usually, the factorization is faster.
Therefore, depending on your use case, either use METIS, or you can compile and
run your code without using METIS.  If you are using METIS on an unsymmetric
case, UMFPACK must form the Matrix $A^{T}A$. This matrix can have many entries
it takes a lot of resources to form it. To avoid such conditions, ParU uses the
ordering strategy \verb'PARU_ORDERING_METIS_GUARD' by default.  This
ordering strategy uses COLAMD instead of METIS in when $A^T A$ is too costly to
construct.

This modified version of METIS is built into CHOLMOD itself, with all
functions renamed, so it does not conflict with a standard METIS library.
The unmodified METIS library can be safely linked with an application that
uses the modified METIS inside CHOLMOD, without any linking conflicts.

The use of OpenMP tasking is optional, but without it, only parallelism within
the BLAS can be exploited (if available).  ParU depends on parallel tasking to
factorize multiple fronts at the same time, and performance will suffer if the
compiler and BLAS library are not suitable for this method.

See \verb'ParU/LICENSE.txt' for the license.  Alternative licenses are also
available; contact the authors for details.

%-------------------------------------------------------------------------------
% References
%-------------------------------------------------------------------------------

\bibliographystyle{plain}
\bibliography{paru_user_guide}
\end{document}
